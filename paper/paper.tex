\documentclass[9pt,letterpaper]{extarticle}
\usepackage{extsizes}
\usepackage[paper=letterpaper,margin=1in]{geometry}

% Table coloring: Must be imported before TikZ
\usepackage[table,x11names,dvipsnames]{xcolor}

\usepackage{xspace,amsmath,amsfonts,amssymb,hyperref,tikz,multirow}
\usepackage{graphicx}
% \graphicspath{{../data/R_plots/}}

\usepackage{parskip}

\newcommand{\todo}[1]{%
	\mbox{}% prevent marginpar from being on previous paragraph
	\marginpar{%
		\colorbox{red!80!black}{\textcolor{white}{to-do}}%
		\vspace*{-22pt}% hack!
	}%
	\textcolor{red}{#1}%
}

\usepackage{algorithmicx}
\usepackage{algorithm}
\usepackage[noend]{algpseudocode}

\usepackage{graphicx}
\usepackage{multicol}

\title{N-dimensional Tic Tac Toe, and Adventure in Modules}
\author{Alex Grasley \and Jeff Young \and Michael McGirr}
\date{}

\begin{document}
	\maketitle
	% \begin{abstract}
  %   Do we want/need an abstract?
	% \end{abstract}

  \section{Introduction}
  \section{Overview of Project}
  % A high-level overview of what your project is about. This should answer the
  % following questions: What is the domain? Who are the users? What kinds of
  % things can those users do with your project?

  Our project initially began as a pokemon simulator - but early on we realized
  that we could make better use of the SML module system by approaching the
  problem of simulating player-based games in a more abstract general way.
  By doing so we could define the basic notion of what a game simulation requires
  and isolate a pattern to follow for any number of games that fit this
  model.
  A game would then be a specific implementation - in our case
  \textit{Tic-tac-toe} - that used this pattern.

  \section{Program description}
  % A list and brief description of all of the signatures, modules, and functors
  % in your project. In the description of signatures and modules, you should
  % highlight especially important types and functions. The goal of this section
  % is to guide the reader through your source code, putting things into their
  % proper context. It isn’t necessary to describe every helper function, but this
  % section should provide a reasonably complete overview of the content of your
  % project.
  
  Our project is separated between the code that describes the game simulation
  and the code that uses this to make a specific \textit{Tic-tac-toe} implementation.
  The game simulation is located in the \texttt{game.sml} file. Likewise, the
  code for the \textit{Tic-tac-toe} implementation is located in
  \texttt{tictactoe.sml} and uses modules from \texttt{matrix.sml}.

  As we touched on before - the notion of a game is generalized in the
  \texttt{game.sml} file. This defines the signatures and functor to run a game.
  The general pattern for a game under our model consists of three purposely
  isolated pieces.
  These are \textit{State}, \textit{Actions}, and \textit{Agents} which are
  each given their own signature in \texttt{game.sml}.
  These are then wrapped together with a functor. The idea is that any game
  consists of a state, a set of actions on that state, and an agents that take
  actions for a state. A specific implementation of a game would use the
  relationship between these three general pieces to define a runnable game with
  its own modules.

  Our implementation of \textit{Tic-tac-toe} uses this game model to operate and
  is primarily defined within \texttt{tictactoe.sml}.
  Here we have two signatures - one that extends state (the \texttt{STATE}
  signature from \texttt{game.sml}) for \textit{Tic-tac-toe}
  called \texttt{TTTSTATE} and another that does the same with action called
  \texttt{TTTACTION} - we don't extend the \texttt{AGENT} signature specifically
  for \textit{Tic-tac-toe}.

  There are many module structures instantiated here for the various kinds of
  state, actions and agents that we eventually want to use -
  such as \texttt{TttState} and \texttt{TttAction} for instance.
  These are bound to functors like \texttt{TttStateFn} which takes
  a module implementing a square matrix module (\texttt{SQUAREMATRIX})
  and returns a module that implements \texttt{TTTSTATE}.
  
  The functor \texttt{TttActionFn} (used by modules structures like
  \texttt{TttAction} and \texttt{Ttt3DAction}) takes a module that implements
  \texttt{TTTSTATE} and gives us a module implementing \texttt{TTTACTION}.
  The functors \texttt{TttRandomAgent} and \texttt{TttHumanAgentFn} take
  modules which implement \texttt{TTTACTION} and return a module that implements
  \texttt{AGENT}.

  We can then instantiate a structure like \texttt{TttExecRandom} that will run
  a game of \textit{Tic-tac-toe} by providing our \texttt{ExecFn} functor from
  \texttt{game.sml} with a module that implements \texttt{AGENT}.
  \texttt{ExecFn} then gives us a module that implements \texttt{EXEC}.
  
  The idea being that we instantiated specific module structures
  (like \texttt{TttState}) for different cases of actions and states that may
  occur in various kinds of \textit{Tic-tac-toe} games.
  We then described functors which linked these structures to the functions
  meant for that specific use and returned a new .
  When provided with the correct input module these functors will give us
  modules that implement that portion of the game pattern - which we then use
  with another structure and another functor for the next modular piece of the
  game.
  % ugh I should probably rewrite that last paragraph

	\section{Design Decisions}
  % (Most important!) Describe in detail 3-5 design decisions you made during your
  % project, with respect to abstraction and/or modularity. For example, what is a
  % piece of information that you chose to hide and why? How did you decide on a
  % particular way of separating your system into modules compared to another. How
  % did you reduce coupling between two modules? These design decisions should
  % emphasize how you applied ideas from the papers we’ve read (and associated
  % discussions) to the design of your project. It might be helpful here to
  % include snippets from old versions (or hypothetical alternative versions) of
  % your project to compare and contrast with the final version.
  \subsection{Creating an Abstract Game Engine}
  \subsection{Higher Ordered Signatures, and the ``Include'' incantation}
  \subsection{Separation of IO, or How I learned to not fight SML in search of Purity}
  \subsection{You can do it in 2-dimensions, but can you do it in n-dimensions!}
  \subsection{The Functor is love, the Functor is life}

\end{document}
