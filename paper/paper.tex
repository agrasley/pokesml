\documentclass[9pt,letterpaper]{extarticle}
\usepackage{extsizes}
\usepackage[paper=letterpaper,margin=1in]{geometry}

% Table coloring: Must be imported before TikZ
\usepackage[table,x11names,dvipsnames]{xcolor}

\usepackage{xspace,amsmath,amsfonts,amssymb,hyperref,tikz,multirow}
\usepackage{graphicx}
% \graphicspath{{../data/R_plots/}}

\usepackage{parskip}

\newcommand{\todo}[1]{%
	\mbox{}% prevent marginpar from being on previous paragraph
	\marginpar{%
		\colorbox{red!80!black}{\textcolor{white}{to-do}}%
		\vspace*{-22pt}% hack!
	}%
	\textcolor{red}{#1}%
}

\usepackage{algorithmicx}
\usepackage{algorithm}
\usepackage[noend]{algpseudocode}

\usepackage{graphicx}
\usepackage{multicol}

\title{N-dimensional Tic Tac Toe, and Adventure in Modules}
\author{Alex Grasley \and Jeff Young \and Michael McGirr}
\date{}

\begin{document}
	\maketitle
	% \begin{abstract}
  %   Do we want/need an abstract?
	% \end{abstract}

  \section{Introduction}
  \section{Overview of Project}
  % A high-level overview of what your project is about. This should answer the
  % following questions: What is the domain? Who are the users? What kinds of
  % things can those users do with your project?
  
  

  \section{Program description}
  % A list and brief description of all of the signatures, modules, and functors
  % in your project. In the description of signatures and modules, you should
  % highlight especially important types and functions. The goal of this section
  % is to guide the reader through your source code, putting things into their
  % proper context. It isn’t necessary to describe every helper function, but this
  % section should provide a reasonably complete overview of the content of your
  % project.



	\section{Design Decisions}
  % (Most important!) Describe in detail 3-5 design decisions you made during your
  % project, with respect to abstraction and/or modularity. For example, what is a
  % piece of information that you chose to hide and why? How did you decide on a
  % particular way of separating your system into modules compared to another. How
  % did you reduce coupling between two modules? These design decisions should
  % emphasize how you applied ideas from the papers we’ve read (and associated
  % discussions) to the design of your project. It might be helpful here to
  % include snippets from old versions (or hypothetical alternative versions) of
  % your project to compare and contrast with the final version.
  \subsection{Creating an Abstract Game Engine}
  \subsection{Higher Ordered Signatures, and the ``Include'' incantation}
  \subsection{Separation of IO, or How I learned to not fight SML in search of Purity}
  \subsection{You can do it in 2-dimensions, but can you do it in n-dimensions!}
  \subsection{The Functor is love, the Functor is life}

\end{document}
